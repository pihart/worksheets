%! Author = avime
%! Name = mathreview
%! Project = LaTeX Hub
%! Date = 5/25/2020

\documentclass[12pt,paper=letter]{article}

\usepackage[page,links,theorem,thmboxed]{avimehra}

\addtolength{\jot}{1em}



\begin{document}

    \title{Math Review}
    \author{Avi Mehra}
    \date{5/25/2020}
    \maketitle


    \section{Basic Operations}


    \section{Sets}
    In Naive Set Theory,
    a set is just a collection of anything.
    The set of all integers between $1$ and $4$, inclusive,
    can be written as
    \[
        \set{1,2,3,4},
    \]
    and the set of the pizza on the counter can be written as
    \[
        \set{\text{the pizza on the counter}}.
    \]

    Sets don't have any order structure;
    the first set may very well be written as
    \[
        \set{3,2,4,1}
    \]
    without changing the meaning whatsoever.

    \subsection{Set terminology}
    \begin{defboxed}
        Anything inside of a set is called an element of the set or a member of the set.
        If $x$ is an element of $S$,
        we write $x\in S$ to show this relation.
        This is sometimes written in reverse as $S\ni x$.
    \end{defboxed}

    \begin{defboxed}
        A set is called finite if it has a natural number of elements.
    \end{defboxed}

    \begin{defboxed}
        For a finite set $S$,
        its cardinality $\abs{S}$ is the number of elements it contains.
    \end{defboxed}

    \begin{defboxed}
        A set is called countable if you can write down a (possibly infinite) list of its elements without missing any.
        The natural numbers are countable, because you can list each natural number just by counting up by 1 from 0.
        All finite sets are countable.
    \end{defboxed}

    \begin{defboxed}
        A set is countably infinite if it is countable and infinite.
    \end{defboxed}

    \begin{defboxed}
        We say a set is uncountable if it is not countable.
        All uncountable sets are infinite.
    \end{defboxed}

    \begin{defboxed}
        For a finite set $F$,
        countably infinite set $C$,
        and uncountable set $U$,
        their cardinalities are ordered as follows
        \[
            \abs{F}<\abs{C}<\abs{U}.
        \]
        That is,
        the cardinality of a finite set is always less than the cardinality of a countably infinite set,
        which is always less than the cardinality of an uncountable set.
    \end{defboxed}

    \subsection{Set relations}

    \begin{defboxed}
        If every element of $S$ is also an element of $T$,
        we write $S\subseteq T$ and say $S$ is a subset of $T$.
        If $S\subseteq T$ and $S\ne T$,
        we write $S\subset T$ and say $S$ is a proper subset of $T$.
    \end{defboxed}
    \begin{defboxed}
        The relationship $S\subseteq T$ is the same as $T\supseteq S$,
        which is read ``T is a superset of S."
        Similarly, $S\subset T$ is the same as $T\supset S$,
        which is read ``T is a proper superset of S."
    \end{defboxed}

    \subsection{Commonly encountered sets}

    \begin{tabular}{|c|c|c|}
        \hline
        Name & Symbol & Expansion\\
        \hline
        Integers & $\Z$ & $\set{\dots, -3, -2, -1, 0, 1, 2, 3, \dots}$\\
        Natural Numbers & $\N$ & $\set{0, 1, 2, 3, 4, 5, 6, \dots}$\\
        Positive Integers & $\Z^+$ & $\set{1, 2, 3, 4, 5, 6, 7, \dots}$\\
        Real Numbers & $\R$ & Any ``decimal" number, uncountable\\
        \hline
    \end{tabular}

    \subsection{Set representations}
    There are, however,
    other ways of expressing sets.
    For example,
    the first set we looked at can be rewritten as
    \[
        \set{n\in\Z\mid 1\leq n\leq 4}.
    \]
    This notation reads ``the set of all integers $n$ where $1\leq n\leq 4$."

    We can use this notation to write the set $S$ of real numbers between 0 (inclusive) and 1 (exclusive):
    \[
        S = \set{x\in\R\mid 0\leq x<1}.
    \]
    However,
    it can also be written as
    \[
        S = [0,1).
    \]
    In this notation,
    we use $[a,b)$ to represent the range $\set{x\in\R\mid a\leq x<b}$ between $a$ and $b$.
    Square brackets represent an inclusive bound while parenthesis represent an exclusive bound.
    We can use also use the exclusive bounds $+-\infty$:
    the set of all real numbers greater than 6 is $(6,\infty)$ and the set of all real numbers less than 2 is $(-\infty,2)$.

    \subsection{Set operations}

    \begin{defboxed}
        Given sets $A$ and $B$,
        the union $A\cup B$ is defined as the set of all elements in either set:
        \[
            A\cup B := \set{s\mid s\in A \text{ or } s\in B}.
        \]
    \end{defboxed}
    \begin{defboxed}
        Given sets $A$ and $B$,
        the intersection $A\cap B$ is defined as the set of all elements common to both sets:
        \[
            A\cap B := \set{s\mid s\in A \text{ and } s\in B}.
        \]
    \end{defboxed}
    \begin{defboxed}
        Given sets $A$ and $B$,
        the set difference (or asymmetric difference) $A\smallsetminus B$ is defined as the set of all elements in $A$ but not in $B$:
        \[
            A\smallsetminus B := \set{s\mid s\in A \text{ and } s\not\in B}.
        \]
    \end{defboxed}
    \begin{defboxed}
        Given sets $A$ and $B$,
        the symmetric difference (or disjunctive union) $A\triangle B$ is defined as the set of all elements in either set but not both:
        \[
            A\triangle B := \set{s\mid s\in A \text{ xor } s\in B} = A\cup B \smallsetminus A\cap B.
        \]
    \end{defboxed}


    \section{Functions}

    \begin{enumerate}
        \item Continuity
        \item Intermediate value theorem
        \item Mean value theorem
        \item Invertible
        \item method for calculating inverses
    \end{enumerate}


    \section{Equations and Equality}
    When you have an equation,
    both the LHS and RHS are equivalent,
    so any function or operation can be applied to both of them while preserving the equality.
    This is because you are \bluebf{applying the function on the \emph{same thing}};
    of course the results will be the same.

    \subsection{expansion, dist property, foil, binomial thm}


    \section{Euclidean geometry}

    \begin{enumerate}
        \item angles
        \subitem right
        \subitem acute
        \subitem obtuse
        \item triangles
        \subitem area
        \subitem perimeter
        \subitem similarity
        \subitem heron's formula
        \subitem centers
        \subsubitem incenter
        \subsubitem circumcenter
        \subitem sum of internal angles
        \subsubitem angle chasing
        \subitem right, acute, obtuse
        \subsubitem pythagorean theorem, pythagorean inequalities
        \subitem scalene, isosceles, equilateral
        \subsubitem isosceles and equilateral area problems
        \item circles
        \subitem area
        \subitem circumference
        \subitem area and circumference of sectors
        \subitem radius is orthogonal to tangent vector
        \subitem angle properties
        \item solids
        \subitem prism
        \subsubitem volume = bh
        \subsubitem surface area
        \subitem cylinders
        \subitem pyramids
        \subsubitem volume = bh/3
        \subsubitem surface area
        \subitem cone
        \subitem cross-sections
        \item transversal theorem
        \item vertical, complementary, supplementary angles
    \end{enumerate}


    \section{solving linear equations}


    \section{systems of linear equations}
    \begin{enumerate}
        \item algebraic solution by substitution
        \item algebraic solution by reduction
    \end{enumerate}


    \section{exponentiation (powers)}

    \subsection{raising to the natural power}

    \subsection{raising to the integer power}

    \subsection{raising to the real power}

    \subsection{nth roots}


    \section{solving polynomial equations}

    \subsection{factoring}

    \subsection{completing the square}

    \subsection{quadratic formula}


    \section{parity behavior of polynomial equations}


    \section{exponentiation (exp)}

    \subsection{logarithm}


    \section{systems of equations}
    \begin{enumerate}
        \item solutions by graphing
        \item analytic (algebraic) solutions
    \end{enumerate}


    \section{inequalities}
    \begin{enumerate}
        \item graphing
        \item solving
    \end{enumerate}


    \section{systems of linear inequalities}
    \begin{enumerate}
        \item graphing
        \item solving
    \end{enumerate}


    \section{probability}
    \begin{enumerate}
        \item union and intersection formulas
        \item independence
    \end{enumerate}

\end{document}
